\documentclass[brudnopis]{xmgr}
% Jeśli nowe rozdziały mają się zaczynać na stronach nieparzystych:
%\documentclass[openright]{xmgr}

% install minted package to highlight source code
% \usepackage{minted}

%\defaultfontfeatures{Scale=MatchLowercase}
%\setmainfont[Numbers=OldStyle,Ligatures=TeX]{Minion Pro}
%\setsansfont[Numbers=OldStyle,Ligatures=TeX]{Myriad Pro}
% for fontspec version < 2.0
% \setmainfont[Numbers=OldStyle,Mapping=tex-text]{Minion Pro}
% \setsansfont[Numbers=OldStyle,Mapping=tex-text]{Myriad Pro}
%\setmonofont[Scale=0.75]{Monaco}
\usepackage{marginnote}
% Opcjonalnie identyfikator dokumentu
% drukowany tylko z włączoną opcją 'brudnopis':
\wersja   {wersja wstępna [\ymdtoday]}

\author   {Adam Makiewicz}
\nralbumu {235281}
\email    {adammak23@gmail.com}


\title    {Generowanie płytek obwodu drukowanego w środowisku Python jako dodatek do programu graficznego Blender}
\date     {2020}
\miejsce  {Gdańsk}

\opiekun  {dr P. Arłukowicz}

% dodatkowe polecenia
%\renewcommand{\filename}[1]{\texttt{#1}}
%\definecolor{stress}{cmyk}{0,1,0.13,0} % RubineRed
%\definecolor{topic}{cmyk}{0.98,0.13,0,0.43} % MidnightBlue

\begin{document}

% streszczenie
\begin{abstract}
Istnieje wiele programów do projektowania płytek drukowanych, jednak żadne z nich, z uwagi na swoje ścisłe zastosowania, nie posiadają odpowiednich narzędzi do zaawansowanego renderowania, animacji i tworzenia szeroko pojętej “sztuki”. Popularny program do tworzenia grafiki 3D - Blender, z uwagi na możliwość rozbudowania go o dodatki jest znakomitym narzędziem mogącym wspomagać ten proces.
\marginnote {opisać więcej? jeżeli tak to każdy rozdział?}
\end{abstract}


% słowa kluczowe
\keywords{wizualizacja, grafika, 3D, Blender, Python, PCB}

% tytuł i spis treści
\maketitle

% wstęp
\introduction

Obwody drukowane czy też inaczej płytki drukowane (zwane dalej "PCB", ang. Printed Circuit Board) to podstawa dla każdego modułu elektronicznego. Dzięki swojej budowie oraz dobranym częściom składowym pozwalają inżynierom z roku na rok konstruować coraz to nowocześniejsze i bardziej funkcjonalne urządzenia. PCB służy przede wszystkim do montowania wszelkich podzespołów elektronicznych oraz zapewnienia im wspólnego stabilnego połączenia.

Tworzenie PCB składa się z trzech głównych etapów: \cite{Abboud}

\begin{itemize}
\item
Logic Design - Stworzenie schematu logiki i reguł projektowych, spis użytych komponentów i ich wzajemnych połączeń
\item
Layout - Zaprojektowanie układu, który decyduje o fizycznym położeniu i połączeniach (tzw.  \emph{routing}) komponentów
\item
Produkcja przemysłowa
\end{itemize}
    
    Najważniejszym punktem projektowania układu jest rozmieszczenie komponentów. Ten proces jest satysfakcjonującym twórczym przedsięwzięciem i prawdopodobnie jednym z najtrudniejszych aspektów procesu projektowania PCB. Wielu inżynierów uważa go za formę sztuki gdyż w przeciwieństwie do schematu, który opiera się tylko na matematyce, jest nieco bardziej płynny i elastyczny. 

\begin{figure}[!tbh]
\centering
\includegraphics[width=0.9\hsize]{fig/hokusai}
\caption{Joel Betancourt znany jako Garabating - "Katsushika Hokusai Electronic Circuit Board"\label{RYS.1}}
\source{https://garabating.com/post/44549621917/katsushika-hokusai-electronic-circuit-board}
\end{figure}

Nie oznacza to jednak pełnej dowolności w projekcie, gdyż należy wziąć pod uwagę mnogość technicznej wiedzy, pomiarów i zależności takich jak: optymalizacja długości ścieżek oraz ich szerokość, ochrona miejsc narażonych na dużą temperaturę, ograniczenia mechaniczne i montażowe, itd. Nawet zastosowanie automatucznego wyznaczania ścieżek do optymalizacji nie zawsze da poprawny rezultat.\footnote{~https://www.autodesk.com/products/eagle/blog/top-10-pcb-component-placement-tips-pcb-beginner/} Z uwagi na ilość i różnorodność ograniczeń nie jest możliwa całkowita automatyzacja sprawdzania poprawności wykonanego projektu, zatem przydatna dla projektanta okazuje się wizualizacja efektu końcowego. Jest ona także niezbędnym elementem procesu marketingowego, logistycznego czy edukacyjnego. Istnieje wiele programów do projektowania PCB jednak żadne z nich, z uwagi na swoje ścisłe zastosowania, nie posiadają odpowiednich narzędzi do zaawansowanego renderowania, animacji i tworzenia szeroko pojętej “sztuki”. Popularny program do tworzenia grafiki 3D - Blender, z uwagi na możliwość rozbudowania go o dodatki jest znakomitym narzędziem mogącym wspomagać ten proces.


% ROZDZIAŁ 1


\chapter{Cel i zakres pracy magisterskiej}

Celem niniejszej pracy jest stworzenie łatwego do rozbudowania i spójnego systemu umożliwiającego import plików projektowych używanych bezpośrednio w przemyśle PCB do programu Blender, następnie interpretację i wyświetlenie pełnowymiarowego modelu 3D płytki drukowanej która powstałaby w procesie produkcji przemysłowej. Dodatek powinien posiadać prosty i przejrzysty interfejs który zapewnia dostęp do wszystkich funkcjonalności, ale nie przytłacza odbiorcy nadmiarem funkcji. Pozwoli to nie tylko osobom technicznym z poza branży grafiki komputerowej na łatwy dostęp do wizualizacji i edycji swoich projektów ale także na łatwiejszą integrację projektów przemysłowych z marketingową i graficzną częścią przemysłu.

\section{Wymagania funkcjonalne}

Zrealizowany system jest dodatkiem (tzw. \emph{add-on}) do programu Blender, kompatybilnym z wersją 2.8 w zwyż. Wybór konkretnie tej wersji programu był podyktowany jego nową odsłoną oferującą między innymi nowy interfejs i API. Addon udostępnia użytkownikowi dodatkowe funkcjonalności z poziomu graficznego interfejsu użytkownika:
\begin{itemize}
\item Wybranie folderu zawierającego wszystkie pliki projektu PCB lub wybranie pojedynczych plików warstw i  plików pozycji elementów (ang. \emph{"Pick And Place"})\footnote{~Więcej o strukturze plików projektowych PCB w punkcie 1.2.3}
\item Wybranie wbudowanej lub własnej biblioteki modeli 3D
\item Wybór końcowej rozdzielczości i miejsca zapisu plików wytworzonych w procesie renderowania
\item Przycisk tworzący model 3D płytki na podstawie wybranych plików
\end{itemize}

\begin{figure}
\centering
\includegraphics[width=0.75\hsize]{fig/addon_1}
\caption{Zainstalowany dodatek w programie Blender 2.82a\label{RYS.2}}
\source{Opracowanie własne}
\end{figure}

\section {Opis technologii wykorzystanych w pracy}

% Python
\subsection{Python}
Python jest językiem programowania wysokiego poziomu, posiadającym aktywną społeczność i nieograniczone możliwości poprzez rozbudowę go o zewnętrzne pakiety.\footnote {~https://www.python.org/about/} API Blendera jest w większości przygotowane do użycia właśnie Pythona i chociaż istnieją ograniczenia tego, co Python może zrobić w Blenderze, jest to jedyne oficjalnie wspierane rozwiązanie dzięki któremu wiele można osiągnąć bez konieczności zagłębiania się w kod C / C++ Blendera.

% Blender
\subsection {Blender 2.8}
Darmowy program \emph{Open-source}\footnote{~https://www.blender.org/about/license/} cechujący się wszechstronnością i możliwością rozbudowania go o dodatkowe biblioteki lub skrypty napisane w języku Python, które poszerzają podstawowe funkcjonalności.

Dodatek do programu Blender różni się od dodatkowej biblioteki Pythona jedynie pewnymi dodatkowymi wymaganiami jak obiekt zawierający metadane, takie jak: tytuł, wersja, kategoria, autor, etc. Określa też minimalną wersję Blendera wymaganą do uruchomienia skryptu. Dodatek jest więc sposobem na enkapsulację modułu Pythona w sposób, który użytkownik może z łatwością wykorzystać.

Blender ma wbudowany interpreter Pythona, który jest ładowany po uruchomieniu programu. Utrudnia to znacząco wykorzystanie automatycznego pobierania i instalowania zależnych od siebie pakietów, co za tym idzie, tworzony w ramach pracy system, aby ułatwić korzystanie z niego, musi być niezależny od zewnętrznych, dynamicznie pobieranych bibliotek.


\subsection{Pliki projektowe PCB\newline(Gerber, Drill, Pick-And-Place)}

% Gerber
Nowoczesne płytki drukowane są projektowane przy pomocy dedykowanego oprogramowania a ostatnim etapem produkcji dla projektanta jest wygenerowanie między innymi plików Gerber.\cite{Khandpur}
Format ten jest otwartym, wektorowym, powszechnie stosowanym standardem ASCII służącym do przesyłania danych projektowych obwodów drukowanych do przemysłu produkcji elektroniki. Wszystkie systemy do projektowania obwodów drukowanych pozwalają na eksport projektu jako pliki Gerber i każde przemysłowe oprogramowanie do ich obróbki potrafi je interpretować, umożliwiając profesjonalistom w dziedzinie PCB bezpieczną i wydajną wymianę danych projektowych.\cite{Williams}

Płytki drukowane mogą być jednostronne (jedna warstwa miedzi), dwustronne (dwie warstwy miedzi po obu stronach warstwy podłoża) lub wielowarstwowe (zewnętrzna i wewnętrzna warstwa miedzi, naprzemiennie z warstwami podłoża).\cite{schroeder}

Pliki Gerber reprezentują między innymi warstwy miedzi, maskę lutowniczą, legendę oraz dane wiercenia i trasy. Dodatkowe atrybuty dostarczają informacji o sposobie montowania, połączeń i nazw poszczególnych elementów. Format pliku Gerber jest prosty, kompaktowy i jednoznaczny. Jest bazowany na języku \emph{G-code}, oznaczanym RS-274. Dzięki zastosowaniu 7-bitowych znaków ASCII jest czytelny dla człowieka i łatwy do debugowania. Obecnie używany od 2014 roku format to tzw. Rozszerzony Gerber (ang. \emph{Extended Gerber}) lub RS-274X.\footnote{~https://www.ucamco.com/en/gerber}

% Drill / NC / Excellon
Kolejnym, generowanym przez projektanta płytki formatem używanym w produkcji są pliki wierceń (ang. \emph{NC / CNC drill}), pierwotnie zaprojektowane przez twórców wiercących i trasujących maszyn CNC jako zastrzeżone, dedykowane formaty wejściowe dla ich urządeń. Znane są pod nazwami takimi jak: Excellon, Hitachi, Sieb \& Meyer, Posalux, itd.\cite{Charras} Wszystkie z pośród tych formatów są podobne, ponieważ opierają się na wspomnianym wcześniej G-code. Rodzaje wierceń w PCB dzielą się na otwory zwykłe - NPTH (ang. \emph{Not Plated Through Hole}) i pokryte miedzią - PTH (ang. \emph{Plated Through Hole}).\cite{voldman} Stosowanie innego standardu nie jest jednak konieczne, jako że z czasem formaty te zmieniły swoje zastosowanie i obecnie powszechnie stosowaną praktyką jest generowanie tych plików w formacie Gerber.

% Placement
Ostatnim i dosyć kluczowym elementem są pliki wskazujące elementy rozmieszczane na płytce. Jest to prosty format nazywany \emph {Pick-And-Place, Placement list, X-Y file, Mount SMD}  i składa się z kilku wartości:
\begin{itemize}
\item \emph{Ref / Designator} - Indeks elementu na płytce i w projekcie
\item \emph{Value} - Wartość elementu (np. pojemność, rezystancja, napięcie)
\item Pozycja podana we współrzędnych kartezjańskich
\item \emph{Footprint / Package} - Nazwa elementu, zazwyczaj zawiera także informację o jego wymiarach
\item Rotacja elementu
\item Informacja czy objekt znajduje się na wierzchniej czy też dolnej stronie płytki
\item Ewentualne komentarze i dodatkowe informacje
\end{itemize}
Plik zawiera opis elementów montowanych za pomocą technologii montażu przewlekanego - THT (eng. \emph{Through-hole technology}) oraz powierzchniowego - SMT (ang. \emph{Surface Mount Technology}), powszechnie stosowanego przemysłowo.\cite {prasad} Nie jest to jednak format ściśle ustandaryzowany i przy masowej produkcji każdy producent musi manualnie zweryfikować opisy elementów. Na szczęście każdy program do tworzenia PCB posiana eksporter do generowania tychże plików. Dodatkowo są one proste w zapisie i z łatwością edytowalne przez człowieka.

% WRL
\subsection{Pliki VRML i X3D}
Format \emph{.wrl} zwany VRML (ang. \emph {Virtual Reality Modeling Language}) powstały w 1994 roku, stał się pierwszym internetowym formatem 3D, został później zastąpiony przez format X3D.\cite{vrml} Jego ówczesna powszechność sprawiła, że wiele programów przemysłowych tworzonych w latach dziewięćdziesiątych posiada bazy modeli w tym właśnie formacie. Więcej o praktycznym wykorzystaniu tego standardu w rozdziale \ref{bazamodeli}.

% specyfikacja formatu VRML:
% http://www.graphics.stanford.edu/courses/cs248-98-fall/Assignments/Assignment3/VRML2_Specification/

\subsection{Środowisko Visual Studio Code}
Visual Studio Code jest to darmowy edytor kodu który według badań przeprowadzanych co roku przez serwis StackOverflow\footnote{https://insights.stackoverflow.com/survey/2018/\#development-environments-and-tools}$^{,}$\footnote{https://insights.stackoverflow.com/survey/2019\#development-environments-and-tools} cieszy się coraz większym uznaniem. Wybór tego środowiska nie był jednak dyktowany jego popularnością lecz nowymi możliwościami otwierającymi się dzięki instalowaniu w nim rozszerzeń.\footnote{https://code.visualstudio.com/docs/editor/extension-gallery} Dodatek stworzony przez Jacques Lucke na potrzeby rozwoju addonów \marginnote{czy mogę tu użyć słowa "addonów" czy jest to potoczny/angielszczyzm?} do programu Blender pozwala między innymi na automatyzację procesu aktualizowania tworzonego addonu przez tworzenie skrótu w wewnętrznych folderach Blendera, co znacznie przyśpiesza pracę nad systemem. Ponadto posiada ułatwienia takie jak debugowanie w konsoli, tworzenie odpowiednich struktur i przydatne komendy. Sposób działania dodatku ma na celu także upewnienie się, że rozszerzenie nie koliduje z innym menedżerem pakietów Pythona.\footnote{https://marketplace.visualstudio.com/items?itemName=JacquesLucke.blender-development} 

\marginnote {co z tymi footnote'ami? zmniejszyć font? czy mogą takie być?}


% ROZDZIAŁ 2

% Czy i gdzie to dać?
% przegląd krytyczny literatury - analiza krytyczna dotychczasowych badań/rozwiązań z zakresu tematu?
% majenkotech i Eagle2Blender oba bardzo specyficzne do gEDA i Eagle nieposiadające interfejsu, niewygodne do użycia, wymagają dotatkowych narzędzi, znikoma dokumentacja
%https://bitbucket.org/hyOzd/pcb2blender/src/master/
% https://github.com/majenkotech/PCB-Blend

\chapter{Architektura zrealizowanego systemu}
W tym rozdziale zostanie omówiony sposób implementacji systemu i jego kluczowe funkcjonalności.

% moduły zewnętrzne pythona: cairo, gerber etc.
% (+ importer?)
\section{Założenia i wymagania projektowe}
Jako, że addon musi posiadać interfejs, musi być podzielony na moduły odpowiedzialne za logikę i renderowanie interfejsu. System został zaimplementowany w oparciu o wzorzec MVVC z uwagi na to, że jest to struktura zalecana i logicznie wynikająca ze sposobu używania API Blendera. Więcej o implementacji wzorca w podrozdziale \ref{mvc}.
System powinien spełniać wszystkie poniższe wymagania.
\begin{itemize}
\item Czytanie, interpretacja plików excellon, gerber

%interpretacja Gerber (RS-274X) i Excellon, interpretacja kształtów i rysowanie za pomoca krzywych, renderowanie obrazu płytki
%PCB Tools currently supports the following file formats:
%    Gerber (RS-274X)
%    Excellon
%with planned support for IPC-2581, ODB++ and more.
%https://pcb-tools.readthedocs.io/en/latest/about.html
\item renderowanie - cairo
\item renderowanie modelu z wymiarów warstwy outline/jakiejkolwiek innej warstwy jeżeli nie ma
\item czytanie pliku placement (.csv), cache, smart search
%(placement files) ładowanie komponentów i linkowanie plików pomysł wzięty z: https://github.com/majenkotech/PCB-Blend

% niedziałający importer w Blenderze przy implementacji w rozdziale 2
\item baza modeli do PickAndPlace (+tworzenie bazy modeli - importer *.wrl) https://kicad.github.io/packages3d/
\item musi w miarę mało ważyć więc baza modeli będzie osobno

%All Parts in this repository are licensed under CC-BY 3.0 http://creativecommons.org/licenses/by/3.0/
Aby zrealizować założenia pracy, wymagana była baza modeli podzespołów elektronicznych montowanych na PCB. Z uwagi na mnogość producentów, rodzajów i typów podzespołów oraz fakt, że każdy program służący do projektowania może oznaczać je inaczej, optymalnym rozwiązaniem wydaje się udostępnienie użytkownikowi podstawowej bazy modeli. Ponadto zastozowanie szukania modeli częściowo dopasowanych nazwą. Z uwagi na fakt, że niemożliwym jest obsługa wszystkich wyjątków, koniecznością staje się umożliwienie użytkownikowi kożystanie z własnych modeli.

% inne bazy modeli:
% BARDZO DUŻO, ale trzeba by bylo pobierać wszystko skryptem
% https://grabcad.com/
% https://www.3dcontentcentral.com/
% https://www.digikey.com/en/resources/3d-models
% https://www.te.com/
% https://www.traceparts.com/en

\end{itemize}

\section {Struktura projektu}
\subsection{przegląd funkcjonalności}
\subsection{Implementacja wzorca projektowego} \label{mvc}
\begin{itemize}
\item omówienie MVVC
\item omówienie pokrótce plików projektu, podział na moduły - interfejs, logika, init
\item klas
\end{itemize}
\subsection{baza modeli} \label{bazamodeli}
importer?




\chapter{Szczegóły implementacyjne systemu}
\section{Podstawowe komponenty}
\section{Funkcjonalność panelu}

\chapter{Podsumowanie}

W efekcie końcowym zostały utworzone \emph{de facto} 2 addony.
Aplikacja,  mimo  iż  jest  dedykowana  dla  projektów stworzonych w gEDA, KiCad,  jest  o  wiele  bardziej uniwersalna, ponieważ w praktyce wystarczy dodać modele, aby uzyskać format możliwy do odczytania. Podsumowując,  w  wyniku  prac  implementacyjnych  udało  się  zrealizować projekt,  który  jest  zgodny  z  przyjętymi  wcześniej  wymaganiami  i  założeniami przedstawionymi  na  etapie  opracowywania  koncepcji  rozwiązania.  Pomijając  jedną dużą modyfikację związaną ze zmianą metodologii postępowania, projekt powstałych klas pozostał w niezmienionej formie. Warto także zaznaczyć, że przyjęta architektura realizuje   wszystkie   postawione   wymagania   funkcjonalne.   Niestety   aplikacja   nie uniknęła wad. Jedną, ale za to dość poważną, jest spore zapotrzebowanie na pamięć przy przetwarzaniu dużych plików. Jednak dla mniejszych zbiorów danych wydajność czasowa jest na akceptowalnym poziomie. W  kolejnym  rozdziale  przedstawione  zostanie  podsumowanie całokształtu prac związanych z analizą oraz implementacją omawianej aplikacji.

\section{Realizacja założonych celów pracy magisterskiej}
\section {Problemy napotkane podczas realizacji systemu}
\section{Możliwości rozwoju systemu}
\section{Wnioski?}



\begin{table}[!htb]
\begin{tabular}{|l|l|l|} \hline
Nazwa & Autor      & Adres URL \\ \hline
\texttt{sablotron} & Ginger Alliance & \url{http://www.gingerall.com} \\ \hline
\texttt{Xt}        & J.~Clark & \url{http://www.jclark.com} \\ \hline
\texttt{4XSLT}     & FourThought & \url{http://www.fourthought.com} \\ \hline
\texttt{Saxon}     & Michael Kay &  \url{http://users.iclway.co.uk/mhkay/saxon} \\ \hline
\texttt{Xalan}     & Apache XML Project & \url{http://xml.apache.org} \\ \hline
\end{tabular}
\caption{Publicznie dostępne procesory XLST\label{zest:proces:xslt}}
\source{Opracowanie własne}
\end{table}


% zakończenie
\summary
Możliwości, jakie stoją przed archiwum prac magisterskich opartych na
XML-u, są ograniczone jedynie czasem, jaki należy poświęcić na pełną
implementację systemu. Nie ma przeszkód technologicznych do stworzenia
co najmniej równie doskonałego repozytorium, jak ma to miejsce w
przypadku ETD. Jeżeli chcemy w pełni uczestniczyć w rozwoju nowej ery
informacji, musimy szczególną uwagę przykładać do odpowiedniej
klasyfikacji i archiwizacji danych. Sądzę, że język XML znacznie to
upraszcza.

% załączniki (opcjonalnie):
%\appendix
%\chapter{Tytuł załącznika 1}
%Treść załącznika 1.


% literatura (obowiązkowo):
\bibliographystyle{unsrt}
\bibliography{literatura}

% spis tabel (jeżeli jest potrzebny):
\listoftables

% spis rysunków (jeżeli jest potrzebny):
\listoffigures

\oswiadczenie

\end{document}
